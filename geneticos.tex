\documentclass{article}
\usepackage[sc]{mathpazo}
\usepackage[utf8]{inputenc}
\linespread{1.05}
\usepackage[spanish]{babel}
\usepackage{enumitem}
\setlist[itemize]{noitemsep}
\usepackage{blindtext}
\usepackage{lettrine}
\usepackage[hmarginratio=1:1,top=32mm,columnsep=20pt]{geometry} 
\usepackage{titling}
\usepackage{abstract} % Allows abstract customization
\usepackage{tipa}
\usepackage{graphicx}
\renewcommand{\abstractnamefont}{\normalfont\bfseries} 
\usepackage{titlesec} % Allows customization of titles
\renewcommand\thesection{\Roman{section}} % Roman numerals for the sections
\renewcommand\thesubsection{\roman{subsection}} % roman numerals for subsections

\setlength{\droptitle}{-4\baselineskip} % Move the title up
\pretitle{\begin{center}\Huge\bfseries} % Article title formatting
\posttitle{\end{center}} % Article title closing formatting
\title{Implementación de Algoritmos Genéticos para Problema de la Cadena más Cercana (CSP)}
\author{
  Diego Carrillo Verduzco \\
  Universidad Nacional Autónoma de México
}
\date{\today}
\renewcommand{\maketitlehookd}{
\begin{abstract}
  Los algoritmos genéticos son heurísticas inspiradas en los mecanísmos de la selección
  natural para resolver problemas de búsqueda y optimización. Con fines didácticos se realizó una
  implementación de esta heurística orientada hacia encontrar soluciones de instancias del problema
  de la cadena más cercana (CSP) con el fin de encontrar su efectividad a través de experimentación.
  Se encontraron resultados adecuados acerca de la heurística y realidades incómodas sobre la generación
  aleatoria de instancias del problema.
\end{abstract}
}
\begin{document}
\maketitle
\section{Introducción}
  El propósito de este documento es presentar una visión general de los algoritmos genéticos, describir
  brevemente su proceso de implementación y discutir su efectividad al utilizarlos para taclear el problema
  de la cadena más cercana. Adicionalmente, se describe este problema y se interroga la fiabilidad de 
  generar instancias de él aleatoriamente para propósitos de experimentación.

\section{La Heurística}
  Un algoritmo genético es una metaheurística inspirada por el proceso de selección natural, utilizada
  para generar soluciones a problemas de búsqueda y optimización.

  Para utilizar un algorítmo genético para encontrar soluciones a un problema, primero es necesario 
  encontrar una representación \textit{cromosómica} o \textit{genotípica} de la solución; típicamente
  esto implica encontrar una representación en cadenas binarias de 1s y 0s, pero la codificación no tiene
  que ser necesariamente de esa forma, basta con que la codificación encapsule apropiadamente las características 
  de la solución.
  
  En segundo lugar, se ha de establecer una función de \textit{aptitud}, que determine la calidad de cierta
  solución. En problemas de optimización, esta aptitud puede ser la función objetivo que se intenta resolver.

  Al iniciar el algoritmo genético, se tiene una población de \textit{individuos} (soluciones candidatas)
  de entre los cuales se debe seleccionar estocásticamente para engendrar una nueva generación de individuos.
  Usualmente, el proceso de selección debe favorecer a los individuos con mayor aptitud, de tal manera que se
  preserven las propiedades de esos individuos; sin embargo se pueden seleccionar diversas formas de seleccionar 
  a los individuos.

  Una vez habiendo seleccionado a los candidatos para engendrar la siguiente generación, se aplican una serie de 
  \textit{operadores genéticos}. Entre estos se encuentran: 
  \begin{enumerate}
    \item \textbf{Recombinación}: En este operador se combina la información de los cromosomas de dos o más
      individuos diferentes, creando por lo menos un nuevo individuo con carácterísticas de los individuos "padres".
      Usualmente, si se ve la representación genotípica de la solución como un arreglo de bits, la recombinación
      se efectúa escogiendo un índice del arreglo y copiando la información del individuo $A$ al nuevo individuo
      hasta ese índice, y después de ese índice se copia la información del individuo $B$ al nuevo individuo.

    \item \textbf{Mutación}: Con este operador se selecciona al azar alguno de los genes de una solución y se cambia
      aleatoriamente a otro valor. Por ejemplo, si la representación genética fuera una cadena de bits, la mutación
      representaría escoger un bit al azar de la cadena y voltearlo a su valor opuesto.
  \end{enumerate}


\section{El Problema}
\section{La Implementación}
\section{La experimentación}
\section{Los Resultados}
\section{Las Conclusiones}
\end{document}
